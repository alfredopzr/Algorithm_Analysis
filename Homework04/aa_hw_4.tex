%%%%%%%%%%%%%%%%%%%%%%%%%%%%%%%%%%%%%%%%%%%%%%%%%%%%%%%%%%%%%%%%%%
%
% Michael Scherger
%
% Analysis of Algorithms
%
% Homework Assignment #4
%
%%%%%%%%%%%%%%%%%%%%%%%%%%%%%%%%%%%%%%%%%%%%%%%%%%%%%%%%%%%%%%%%%%
%%%%%%%%%%%%%%%%%%%%%%%%%%%%%%%%%%%%%%%%%%%%%%%%%%%%%%%%%%%%%%%%%%
%
% Score Card and Answer Sheets
%
%%%%%%%%%%%%%%%%%%%%%%%%%%%%%%%%%%%%%%%%%%%%%%%%%%%%%%%%%%%%%%%%%%
\documentclass[11pt,addpoints]{exam}
\usepackage{listings}
\usepackage{color}
\usepackage{tcucosc}
\usepackage{clrscode3e}

\definecolor{dkgreen}{rgb}{0,0.6,0}
\definecolor{gray}{rgb}{0.5,0.5,0.5}
\definecolor{mauve}{rgb}{0.58,0,0.82}

\lstset{frame=tb,
	language=Java,
	aboveskip=3mm,
	belowskip=3mm,
	showstringspaces=false,
	columns=flexible,
	basicstyle={\small\ttfamily},
	numbers=none,
	numberstyle=\tiny\color{gray},
	keywordstyle=\color{blue},
	commentstyle=\color{dkgreen},
	stringstyle=\color{mauve},
	breaklines=true,
	breakatwhitespace=true,
	tabsize=3
}
%%%%%%%%%%%%%%%%%%%%%%%%%%%%%%%%%%%%%%%%%%%%%%%%%%%%%%%%%%
%%%
%%% Set Page Size
%%%
%%%%%%%%%%%%%%%%%%%%%%%%%%%%%%%%%%%%%%%%%%%%%%%%%%%%%%%%%%


%%%%%%%%%%%%%%%%%%%%%%%%%%%%%%%%%%%%%%%%%%%%%%%%%%%%%%%%%
%%%
%%% Renew Commands
%%%
%%%%%%%%%%%%%%%%%%%%%%%%%%%%%%%%%%%%%%%%%%%%%%%%%%%%%%%%%%
\renewcommand\refname{}					% for bibliography


%%%%%%%%%%%%%%%%%%%%%%%%%%%%%%%%%%%%%%%%%%%%%%%%%%%%%%%%%%%%%%%%%%
%
% Begin Document
%
%%%%%%%%%%%%%%%%%%%%%%%%%%%%%%%%%%%%%%%%%%%%%%%%%%%%%%%%%%%%%%%%%%
\begin{document}
\pagestyle{empty}

\noindent{\large\bfseries Name: Afredo Perez{\hrulefill}}\\
\noindent{\large\bfseries COSC 40403 - Analysis of Algorithms: Fall 2020: Homework 4}\\
\noindent{\large\bfseries Due: 23:30 on 9/14 }


%%%%%%%%%%%%%%%%%%%%%%%%%%%%%%%%%%%%%%%%%%%%%%%%%%%%%%%%%%%%%%%%%%
%
% Score Card and Answer Sheets
%
% Comment out one-or-the-other to show or not-show the answers.
%
%%%%%%%%%%%%%%%%%%%%%%%%%%%%%%%%%%%%%%%%%%%%%%%%%%%%%%%%%%%%%%%%%%
\printanswers
%\noprintanswers


%%%%%%%%%%%%%%%%%%%%%%%%%%%%%%%%%%%%%%%%%%%%%%%%%%%%%%%%%%%%%%%%%%
%
% Score Card
%
%%%%%%%%%%%%%%%%%%%%%%%%%%%%%%%%%%%%%%%%%%%%%%%%%%%%%%%%%%%%%%%%%%
%\ifprintanswers
%\noindent
%\begin{center}
%	\gradetable[v][questions]
%\end{center}
%%\newpage
%\bigskip
%\fi


Some of the questions below require you to draw a heap or to trace through a heap algorithm on a specific array.  You may do so using \LaTeX packages that you can research and learn how to use.  You may also use a computer drawing program (i.e. PowerPoint or similar), create your diagrams, and export them as a PDF file (I find that it also helps to trim the PDF file using graphics software.)  Then you can import the PDF file into your solution.
\begin{questions}
%%%%%%%%%%%%%%%%%%%%%%%%%%%%%%%%%%%%%%%%%%%%%%%%%%%%%%%%%%%%%%%%%%
%
%%%%%%%%%%%%%%%%%%%%%%%%%%%%%%%%%%%%%%%%%%%%%%%%%%%%%%%%%%%%%%%%%%
\question[5]
What are the $\floor{n/2}$ minimum and maximum number of elements in a heap of height $h$?
\newline

%At most, it is a perfect binary tree, which is $2^(h+1) − 1$; at least, there is only one at the last level, which is $2^h$.
\begin{solutionorbox}
At most, it is a perfect binary tree, which is $2^(h+1) - 1$; at least, there is ony one at the last level, which is $2^h$.
\end{solutionorbox}


\ifprintanswers
\newpage
\else
\bigskip
\fi


%%%%%%%%%%%%%%%%%%%%%%%%%%%%%%%%%%%%%%%%%%%%%%%%%%%%%%%%%%%%%%%%%%
%
%%%%%%%%%%%%%%%%%%%%%%%%%%%%%%%%%%%%%%%%%%%%%%%%%%%%%%%%%%%%%%%%%%
\question[5]
Show that an $n$-element heap has height $\floor{lg n}$.

\begin{solutionorbox}
SOLUTION TO BE ADDED
\end{solutionorbox}

\ifprintanswers
\newpage
\else
\bigskip
\fi


%%%%%%%%%%%%%%%%%%%%%%%%%%%%%%%%%%%%%%%%%%%%%%%%%%%%%%%%%%%%%%%%%%
%
%%%%%%%%%%%%%%%%%%%%%%%%%%%%%%%%%%%%%%%%%%%%%%%%%%%%%%%%%%%%%%%%%%
\question[5]
Is the array with values $\left<23, 17, 14, 6, 13, 10, 1, 5, 7, 12\right>$ a max-heap?  Show your work by using computer software to draw the heap.

\begin{solutionorbox}
	\newline
	In a max-heap every node should be greater than its child, node with value of 6 is less than its child which is node with a value of 7. Hence it is not a max heap.
\begin{verbatim}
def extractMax(self):  
     popped = self.Heap[self.FRONT]
     self.Heap[self.FRONT] = self.Heap[self.size]
     self.size -= 1
     self.maxHeapify(self.FRONT)          
     return popped

\end{verbatim}
\end{solutionorbox}
\ifprintanswers
\newpage
\else
\bigskip
\fi


%%%%%%%%%%%%%%%%%%%%%%%%%%%%%%%%%%%%%%%%%%%%%%%%%%%%%%%%%%%%%%%%%%
%
%%%%%%%%%%%%%%%%%%%%%%%%%%%%%%%%%%%%%%%%%%%%%%%%%%%%%%%%%%%%%%%%%%
\question[10]
Illustrate (using computer software) the operation of $\proc{Max-Heapify}{(A,3)}$ on the array $A = \left< 27, 17, 3, 16, 13, 10, 1, 5, 7, 12, 4, 8, 9, 0\right>$.  Show how a heap element will trickle down the heap level-by-level.

\begin{center}
	\includegraphics[width=\linewidth]{2a}
\end{center}

\ifprintanswers
\newpage
\else
\bigskip
\fi


%%%%%%%%%%%%%%%%%%%%%%%%%%%%%%%%%%%%%%%%%%%%%%%%%%%%%%%%%%%%%%%%%%
%
%%%%%%%%%%%%%%%%%%%%%%%%%%%%%%%%%%%%%%%%%%%%%%%%%%%%%%%%%%%%%%%%%%
\question[5]
Starting with the procedure $\proc{Max-Heapify}{(A,i)}$, write pseudocode for the procedure $\proc{Min-Heapify}{(A,i)}$, which performs the corresponding manipulation on a min-heap.  How does the running time of $\proc{Min-Heapify}$ compare to that of $\proc{Max-Heapify}$?

\begin{lstlisting}
MIN-HEAPIFY(A, i):
	l <- LEFT(i)
	r <- RIGHT(i)
	smallest <- i
	if l <= heap-size[A] and A[l] < A[i]:
		then smallest <- l
	if r <= heap-size[A] and A[r] < A[smallest]:
		then smallest <- r
	if smallest != i:
		then swap(A[i], A[smallest])
			MIN-HEAPIFY(A, smallest)
\end{lstlisting}
\begin{center}
	The running time is the same. Actually, the algorithm is the same with the exceptions of two comparisons and some names.
\end{center}

\ifprintanswers
\newpage
\else
\bigskip
\fi
	
	
%%%%%%%%%%%%%%%%%%%%%%%%%%%%%%%%%%%%%%%%%%%%%%%%%%%%%%%%%%%%%%%%%%
%
%%%%%%%%%%%%%%%%%%%%%%%%%%%%%%%%%%%%%%%%%%%%%%%%%%%%%%%%%%%%%%%%%%
\question[10]
Illustrate (using computer software) the operation of $\proc{Build-Max-Heap}$ on the array $A = \left< 5, 3, 17, 10, 84, 19, 6, 22, 9\right>$.  Each diagram should show the placement of an array element into its final position in the heap.  

\begin{center}
	Illustration attached in file
\end{center}

\ifprintanswers
\newpage
\else
\bigskip
\fi
	
	
%%%%%%%%%%%%%%%%%%%%%%%%%%%%%%%%%%%%%%%%%%%%%%%%%%%%%%%%%%%%%%%%%%
%
%%%%%%%%%%%%%%%%%%%%%%%%%%%%%%%%%%%%%%%%%%%%%%%%%%%%%%%%%%%%%%%%%%
\question[5]
Why do we want the loop index $i$ in line 3 of $\proc{Build-Max-Heap}$ to decrease from $\floor{A.length/2}$ to 1 rather than increase from 1 to $\floor{A.length/2}$?

\begin{center}
	If we start from 1, because its subtree is not a maximum heap, we can't follow this order.
\end{center}

\ifprintanswers
\newpage
\else
\bigskip
\fi
	
	
%%%%%%%%%%%%%%%%%%%%%%%%%%%%%%%%%%%%%%%%%%%%%%%%%%%%%%%%%%%%%%%%%%
%
%%%%%%%%%%%%%%%%%%%%%%%%%%%%%%%%%%%%%%%%%%%%%%%%%%%%%%%%%%%%%%%%%%
\question[10]
Illustrate (using computer software) the operation of $\proc{Heapsort}$ on the array $A = \left< 5, 13, 2, 25, 7, 17, 20, 8, 4 \right>$.  Show the original input array, the heap after $\proc{Build-Heap}$, and then after each call to $\proc{Build-Max-Heap}$.

\begin{center}	
	Illustration attached in file
\end{center}

\ifprintanswers
\newpage
\else
\bigskip
\fi
	
	
%%%%%%%%%%%%%%%%%%%%%%%%%%%%%%%%%%%%%%%%%%%%%%%%%%%%%%%%%%%%%%%%%%
%
%%%%%%%%%%%%%%%%%%%%%%%%%%%%%%%%%%%%%%%%%%%%%%%%%%%%%%%%%%%%%%%%%%
\question[5]
What is the running time of $\proc{Heapsort}$ on an array $A$ of length $n$ that is already sorted in increasing order?  What about decreasing order?

\begin{center}	
If the array is in either ascending or descending order we have a run time of O(n lg n).
\end{center}


%%%%%%%%%%%%%%%%%%%%%%%%%%%%%%%%%%%%%%%%%%%%%%%%%%%%%%%%%%%%%%%%%%
%
%%%%%%%%%%%%%%%%%%%%%%%%%%%%%%%%%%%%%%%%%%%%%%%%%%%%%%%%%%%%%%%%%%
\end{questions}
\end{document}
