%%%%%%%%%%%%%%%%%%%%%%%%%%%%%%%%%%%%%%%%%%%%%%%%%%%%%%%%%%%%%%%%%%
%
% Michael Scherger
%
% Analysis of Algorithms
%
% Homework Assignment #1
%
%%%%%%%%%%%%%%%%%%%%%%%%%%%%%%%%%%%%%%%%%%%%%%%%%%%%%%%%%%%%%%%%%%
%%%%%%%%%%%%%%%%%%%%%%%%%%%%%%%%%%%%%%%%%%%%%%%%%%%%%%%%%%%%%%%%%%
%
% Score Card and Answer Sheets
%
%%%%%%%%%%%%%%%%%%%%%%%%%%%%%%%%%%%%%%%%%%%%%%%%%%%%%%%%%%%%%%%%%%
\documentclass[11pt,addpoints]{exam}
\usepackage{clrscode3e}
\usepackage{tcucosc}
\usepackage{geometry}
%\sepackage[addpoints]{exam}
\usepackage{ulem}
\definecolor{dkgreen}{rgb}{0,0.6,0}
\definecolor{gray}{rgb}{0.5,0.5,0.5}
\definecolor{mauve}{rgb}{0.58,0,0.82}

\lstset{frame=tb,
	language=Java,
	aboveskip=3mm,
	belowskip=3mm,
	showstringspaces=false,
	columns=flexible,
	basicstyle={\small\ttfamily},
	numbers=none,
	numberstyle=\tiny\color{gray},
	keywordstyle=\color{blue},
	commentstyle=\color{dkgreen},
	stringstyle=\color{mauve},
	breaklines=true,
	breakatwhitespace=true,
	tabsize=3
}


%%%%%%%%%%%%%%%%%%%%%%%%%%%%%%%%%%%%%%%%%%%%%%%%%%%%%%%%%%
%%%
%%% Set Page Size
%%%
%%%%%%%%%%%%%%%%%%%%%%%%%%%%%%%%%%%%%%%%%%%%%%%%%%%%%%%%%%
\geometry{hmargin={1.0in,1.0in},vmargin={1.0in,1.0in}}


%%%%%%%%%%%%%%%%%%%%%%%%%%%%%%%%%%%%%%%%%%%%%%%%%%%%%%%%%%%
%%%
%%% Fancy Header
%%%
%%%%%%%%%%%%%%%%%%%%%%%%%%%%%%%%%%%%%%%%%%%%%%%%%%%%%%%%%%


%%%%%%%%%%%%%%%%%%%%%%%%%%%%%%%%%%%%%%%%%%%%%%%%%%%%%%%%%
%%%
%%% Renew Commands
%%%
%%%%%%%%%%%%%%%%%%%%%%%%%%%%%%%%%%%%%%%%%%%%%%%%%%%%%%%%%%
\renewcommand\refname{}					% for bibliography


%%%%%%%%%%%%%%%%%%%%%%%%%%%%%%%%%%%%%%%%%%%%%%%%%%%%%%%%%%%%%%%%%%
%
% Begin Document
%
%%%%%%%%%%%%%%%%%%%%%%%%%%%%%%%%%%%%%%%%%%%%%%%%%%%%%%%%%%%%%%%%%%
\begin{document}
\pagestyle{empty}

\noindent{\large\bfseries Name: ALFREDO PEREZ{\hrulefill}}\\
\noindent{\large\bfseries COSC 40403 - Analysis of Algorithms: Fall 2020: Homework 1}\\
\noindent{\large\bfseries Due: 23:30 on 8/24 }


%%%%%%%%%%%%%%%%%%%%%%%%%%%%%%%%%%%%%%%%%%%%%%%%%%%%%%%%%%%%%%%%%%
%
% Score Card and Answer Sheets
%
% Comment out one-or-the-other to show or not-show the answers.
%
%%%%%%%%%%%%%%%%%%%%%%%%%%%%%%%%%%%%%%%%%%%%%%%%%%%%%%%%%%%%%%%%%%
\printanswers
%\noprintanswers


%%%%%%%%%%%%%%%%%%%%%%%%%%%%%%%%%%%%%%%%%%%%%%%%%%%%%%%%%%%%%%%%%%
%
% Score Card
%
%%%%%%%%%%%%%%%%%%%%%%%%%%%%%%%%%%%%%%%%%%%%%%%%%%%%%%%%%%%%%%%%%%
\ifprintanswers
\noindent
\begin{center}
	\gradetable[v][questions]
\end{center}
%\newpage
\bigskip
\fi

\noindent For questions 1-3, use the following Node definition.  This Node definition is written as a  Python class.
\begin{verbatim}
class Node:
    def __init__(self, data):
        self.data = data
        self.next = None
\end{verbatim}



\begin{questions}
%%%%%%%%%%%%%%%%%%%%%%%%%%%%%%%%%%%%%%%%%%%%%%%%%%%%%%%%%%%%%%%%%%
% Question 
%%%%%%%%%%%%%%%%%%%%%%%%%%%%%%%%%%%%%%%%%%%%%%%%%%%%%%%%%%%%%%%%%%
\question[10]
Write a function (Python) to concatenate two linked lists.  Given lists $l1 = \langle 2, 3, 1 \rangle$ and $l2 = \langle 4, 5 \rangle$, after return from \proc{list\_concatenate}($l1$, $l2$) the list $l1$ should be changed to $l1 = \langle 2, 3, 1, 4, 5 \rangle$.  Your function should not change $l2$ and should not directly link nodes from $l1$ to $l2$ (i.e. the nodes inserted into $l1$ should be copies of the nodes in $l2$).

\begin{solutionorbox}
	\newline
	\begin{lstlisting}
		// Concatenate Two Linked Lists
		
		def copy(self):
			if self.next is None:
				return Cell(self.data)
			else:
				return Cell(self.data, self.next.copy()
		
		def list_concatenate(l1, l2):
			new = l1.copy()
		
			# find the end of the copy
			last = new
			while last.next is not None:
				last = last.next
			# append a copy of the other list
			last.next = l2.copy()
		
	\end{lstlisting}
\end{solutionorbox}

\ifprintanswers
\newpage
\else
\bigskip
\fi


%%%%%%%%%%%%%%%%%%%%%%%%%%%%%%%%%%%%%%%%%%%%%%%%%%%%%%%%%%%%%%%%%%
% Question 
%%%%%%%%%%%%%%%%%%%%%%%%%%%%%%%%%%%%%%%%%%%%%%%%%%%%%%%%%%%%%%%%%%
\question[10]
Write a function to insert a number as the new $i^{th}$ node of a linked list.  Nodes initially in positions $i$, $i+1$, $\dots$, $n$ shoudl be shifted to positions $i+1$, $i+2$, \dots, $n+1$.  Thus, the length of the list will increase by 1.  If the original list contains fewer than $i-1$ nodes, then the number should be inserted at the end of the list.

\begin{solutionorbox}
	\begin{lstlisting}
		def InsertNth(head, data, position):
			start = head
			if position == 0:
				return Node(data, head)
			while position > 1:
				head = head.next
				position -= 1
			head.next = Node(data, head.next)
			return start
	\end{lstlisting}
\end{solutionorbox}

\ifprintanswers
\newpage
\else
\bigskip
\fi


%%%%%%%%%%%%%%%%%%%%%%%%%%%%%%%%%%%%%%%%%%%%%%%%%%%%%%%%%%%%%%%%%%
% Question 
%%%%%%%%%%%%%%%%%%%%%%%%%%%%%%%%%%%%%%%%%%%%%%%%%%%%%%%%%%%%%%%%%%
\question[10]
Write a function to remove duplicate entries in a linked list.  For example, given the list $\langle 5, 2, 2, 5, 3, 9, 2 \rangle$ as input, your function should change the list so that on return from the function it contains $\langle 5, 3, 9, 2 \rangle$.

\begin{solutionorbox}
	\begin{lstlisting}
		def removeDuplicates(self): 
			temp = self.head 
			if temp is None: 
				return
			while temp.next is not None: 
				if temp.data == temp.next.data: 
					new = temp.next.next
					temp.next = None
					temp.next = new 
				else: 
					temp = temp.next
			return self.head
	\end{lstlisting}
\end{solutionorbox}

\ifprintanswers
\newpage
\else
\bigskip
\fi


%%%%%%%%%%%%%%%%%%%%%%%%%%%%%%%%%%%%%%%%%%%%%%%%%%%%%%%%%%%%%%%%%%
% Question 
%%%%%%%%%%%%%%%%%%%%%%%%%%%%%%%%%%%%%%%%%%%%%%%%%%%%%%%%%%%%%%%%%%
\question[10]
(Problem 1.1 on page 22)  Decide whether you think the following statement is true or false.  If it is true, give a short explanation.  If it is false, give a counterexample.\\

\textit{True or false?  In every instance of the Stable Matching Problem, there is a stable matching containing a part $(m,w)$ such that $m$ is ranked first on the preference list of $w$ and $w$ is ranked first on the preference list of $m$.}

\begin{solutionorbox}
True, in every instance of the Stable Matching Problem, there is a stable matching containing a part (m, w) such that m is ranked first on the preference list of w and w is ranked first on the preference list of m.
\end{solutionorbox}

\ifprintanswers
\newpage
\else
\bigskip
\fi


%%%%%%%%%%%%%%%%%%%%%%%%%%%%%%%%%%%%%%%%%%%%%%%%%%%%%%%%%%%%%%%%%%
% Question 
%%%%%%%%%%%%%%%%%%%%%%%%%%%%%%%%%%%%%%%%%%%%%%%%%%%%%%%%%%%%%%%%%%
\question[10]
(Problem 1.3 on page 22-23)  See the textbook for a description of the problem.

\begin{solutionorbox}
	The statement is True. In the stable matching problem the couples are stable when all couples find the other that is equal in its preference. This would mean that when m and w rank first on their preference list they are stable, which is possible in every instance.
\end{solutionorbox}

\ifprintanswers
\newpage
\else
\bigskip
\fi


%%%%%%%%%%%%%%%%%%%%%%%%%%%%%%%%%%%%%%%%%%%%%%%%%%%%%%%%%%%%%%%%%%
%
%%%%%%%%%%%%%%%%%%%%%%%%%%%%%%%%%%%%%%%%%%%%%%%%%%%%%%%%%%%%%%%%%%
\end{questions}
\end{document}
